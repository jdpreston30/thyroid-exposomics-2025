\newpage

\markright{\textit{Supplementary Table 1}}

\begin{landscape}

\noindent{\fontsize{12}{14.4}\selectfont\textbf{SUPPLEMENTARY TABLE 1}}

\noindent{\fontsize{10}{12}\selectfont\textbf{The full library of xenobiotic chemicals employed for chemical annotation and identification.} The library of 710 confirmed xenobiotic chemicals employed for chemical identification. All chemicals besides those which were purchased as part of a mixture with variable concentrations were present in pooled reference plasma at a concentration of 0·47 ng/mL. There are 710 total unique chemicals (i.e., unique CAS numbers), but for some chemicals, there are multiple fragments and/or retention times from different standards used for annotation and identification, thus resulting in 892 total rows in the table. The column "Target RT (min.)" refers to the library retention time for the given chemical. The individual mz columns indicate typical fragments observed for the given chemical. \textit{*Both an endogenous and exogenous (xenobiotic) chemical; $^\dagger$Additional fragments used for validation besides those listed in mz0--mz3.}}

\input{/Users/JoshsMacbook2015/Desktop/thyroid-exposomics-2025/Supplementary/Components/Tables/ST1.tex}

\end{landscape}

\newpage

\markright{\textit{Supplementary Table 2}}

\begin{landscape}

\noindent{\fontsize{12}{14.4}\selectfont\textbf{SUPPLEMENTARY TABLE 2}}

\noindent{\fontsize{10}{12}\selectfont\textbf{Metadata for all chemicals annotated in samples.} \textit{*Both an endogenous and exogenous (xenobiotic) chemical.}}

\fontsize{8.0pt}{9.6pt}\selectfont
\begin{longtable}{>{\raggedright\arraybackslash}p{8cm}>\centering p{2cm}>\centering p{1.75cm}>\centering p{1.75cm}>{\raggedright\arraybackslash}p{9cm}}
\toprule[0.5pt]
\rule{0pt}{12pt}\raisebox{-0.5\height}{\fontsize{10pt}{12pt}\selectfont\textbf{GROUP: Class:} \textit{\underline{Subclass}}: Name} & \raisebox{-0.5\height}{\fontsize{10pt}{12pt}\selectfont\textbf{CAS}} & \raisebox{-0.5\height}{\fontsize{10pt}{12pt}\selectfont\textbf{\shortstack{Potential\\\\EDC}}} & \raisebox{-0.5\height}{\fontsize{10pt}{12pt}\selectfont\textbf{\shortstack{IARC\\\\Group}}} & \raisebox{-0.5\height}{\fontsize{10pt}{12pt}\selectfont\textbf{Superclass: Class}} \\
\midrule[0.5pt]\addlinespace[2.5pt]
\addlinespace[2.5pt]
\endfirsthead
\toprule[0.5pt]
\rule{0pt}{12pt}\raisebox{-0.5\height}{\fontsize{10pt}{12pt}\selectfont\textbf{GROUP: Class:} \textit{\underline{Subclass}}: Name} & \raisebox{-0.5\height}{\fontsize{10pt}{12pt}\selectfont\textbf{CAS}} & \raisebox{-0.5\height}{\fontsize{10pt}{12pt}\selectfont\textbf{\shortstack{Potential\\\\EDC}}} & \raisebox{-0.5\height}{\fontsize{10pt}{12pt}\selectfont\textbf{\shortstack{IARC\\\\Group}}} & \raisebox{-0.5\height}{\fontsize{10pt}{12pt}\selectfont\textbf{Superclass: Class}} \\
\midrule[0.5pt]
\addlinespace[2.5pt]
\endhead
\bottomrule
\endfoot
{\fontsize{10pt}{12pt}\selectfont\textbf{AGROCHEMICALS}} &  &  &  &  \\ 
\hspace*{0.2cm}\textbf{Fungicides} &  &  &  &  \\ 
\hspace*{0.4cm}\textit{\underline{Anilinopyrimidine}} &  &  &  &  \\ 
\hspace*{0.6cm}Cyprodinil & 121552-61-2 & $\checkmark$ & – & Benzenoids: Benzene and substituted derivatives \\ 
\hspace*{0.6cm}Pyrimethanil & 53112-28-0 & $\checkmark$ & – & Benzenoids: Benzene and substituted derivatives \\ 
\hspace*{0.4cm}\textit{\underline{Azole}} &  &  &  &  \\ 
\hspace*{0.6cm}Fluquinconazole & 136426-54-5 & – & – & Organoheterocyclic compounds: Diazanaphthalenes \\ 
\hspace*{0.6cm}Flutriafol & 76674-21-0 & – & – & Benzenoids: Benzene and substituted derivatives \\ 
\hspace*{0.6cm}Penconazole & 66246-88-6 & $\checkmark$ & – & Benzenoids: Benzene and substituted derivatives \\ 
\hspace*{0.6cm}Prochloraz & 67747-09-5 & $\checkmark$ & – & Benzenoids: Phenol ethers \\ 
\hspace*{0.6cm}Tebuconazole & 107534-96-3 & $\checkmark$ & – & Benzenoids: Benzene and substituted derivatives \\ 
\hspace*{0.6cm}Triadimefon & 43121-43-3 & $\checkmark$ & – & Benzenoids: Phenol ethers \\ 
\hspace*{0.4cm}\textit{\underline{Organochlorine}} &  &  &  &  \\ 
\hspace*{0.6cm}Hexachlorobenzene & 118-74-1 & $\checkmark$ & 2B & Benzenoids: Benzene and substituted derivatives \\ 
\hspace*{0.6cm}Pentachloronitrobenzene & 82-68-8 & – & 3 & Benzenoids: Benzene and substituted derivatives \\ 
\hspace*{0.4cm}\textit{\underline{Phthalimide}} &  &  &  &  \\ 
\hspace*{0.6cm}Captan & 133-06-2 & $\checkmark$ & 3 & Organoheterocyclic compounds: Isoindoles and derivatives \\ 
\hspace*{0.6cm}Folpet & 133-07-3 & $\checkmark$ & – & Organoheterocyclic compounds: Isoindoles and derivatives \\ 
\hspace*{0.4cm}\textit{\underline{Sulfamide}} &  &  &  &  \\ 
\hspace*{0.6cm}Dichlofluanid & 1085-98-9 & $\checkmark$ & – & Benzenoids: Benzene and substituted derivatives \\ 
\hspace*{0.6cm}Tolylfluanid & 731-27-1 & $\checkmark$ & – & Benzenoids: Benzene and substituted derivatives \\ 
\hspace*{0.4cm}\textit{\underline{Other}} &  &  &  &  \\ 
\hspace*{0.6cm}Bupirimate & 41483-43-6 & $\checkmark$ & – & Organoheterocyclic compounds: Diazines \\ 
\hspace*{0.6cm}Fludioxonil & 131341-86-1 & $\checkmark$ & – & Organoheterocyclic compounds: Benzodioxoles \\ 
\hspace*{0.6cm}Metalaxyl & 57837-19-1 & $\checkmark$ & – & Organic acids and derivatives: Carboxylic acids and derivatives \\ 
\hspace*{0.6cm}Procymidone & 32809-16-8 & $\checkmark$ & – & Organoheterocyclic compounds: Piperidines \\ 
\hspace*{0.6cm}Pyrazophos & 13457-18-6 & – & – & Organic acids and derivatives: Organic thiophosphoric acids and derivatives \\ 
\hspace*{0.6cm}Vinclozolin & 50471-44-8 & $\checkmark$ & – & Benzenoids: Benzene and substituted derivatives \\ 
 &  &  &  &  \\ 
\hspace*{0.2cm}\textbf{Herbicides} &  &  &  &  \\ 
\hspace*{0.4cm}\textit{\underline{Amide}} &  &  &  &  \\ 
\hspace*{0.6cm}Diphenamid & 957-51-7 & – & – & Benzenoids: Benzene and substituted derivatives \\ 
\hspace*{0.6cm}Napropamide & 15299-99-7 & $\checkmark$ & – & Benzenoids: Naphthalenes \\ 
\hspace*{0.6cm}Pronamide & 23950-58-5 & $\checkmark$ & – & Benzenoids: Benzene and substituted derivatives \\ 
\hspace*{0.4cm}\textit{\underline{Chloroacetanilide}} &  &  &  &  \\ 
\hspace*{0.6cm}Acetochlor & 34256-82-1 & $\checkmark$ & – & Benzenoids: Benzene and substituted derivatives \\ 
\hspace*{0.6cm}Alachlor & 15972-60-8 & $\checkmark$ & – & Benzenoids: Benzene and substituted derivatives \\ 
\hspace*{0.6cm}Allidochlor & 93-71-0 & – & – & Organic acids and derivatives: Carboxylic acids and derivatives \\ 
\hspace*{0.6cm}Dimethachlor & 50563-36-5 & – & – & Benzenoids: Benzene and substituted derivatives \\ 
\hspace*{0.6cm}Metolachlor & 51218-45-2 & – & – & Benzenoids: Benzene and substituted derivatives \\ 
\hspace*{0.6cm}Pretilachlor & 51218-49-6 & – & – & Benzenoids: Benzene and substituted derivatives \\ 
\hspace*{0.6cm}Propachlor & 1918-16-7 & – & – & Benzenoids: Benzene and substituted derivatives \\ 
\hspace*{0.6cm}Propisochlor & 86763-47-5 & $\checkmark$ & – & Benzenoids: Benzene and substituted derivatives \\ 
\hspace*{0.4cm}\textit{\underline{Dinitroaniline}} &  &  &  &  \\ 
\hspace*{0.6cm}Nitralin & 4726-14-1 & – & – & Benzenoids: Benzene and substituted derivatives \\ 
\hspace*{0.6cm}Pendimethalin & 40487-42-1 & $\checkmark$ & – & Benzenoids: Benzene and substituted derivatives \\ 
\hspace*{0.4cm}\textit{\underline{Dinitrophenol}} &  &  &  &  \\ 
\hspace*{0.6cm}2-Methyl-4,6-dinitrophenol & 534-52-1 & – & – & Benzenoids: Phenols \\ 
\hspace*{0.6cm}Dinoseb & 88-85-7 & $\checkmark$ & – & Benzenoids: Phenols \\ 
\hspace*{0.4cm}\textit{\underline{Phenoxy}} &  &  &  &  \\ 
\hspace*{0.6cm}2-Methyl-4-chlorophenoxyacetic acid & 94-74-6 & – & 2B & Benzenoids: Benzene and substituted derivatives \\ 
\hspace*{0.6cm}Fenoprop & 93-72-1 & – & – & Benzenoids: Benzene and substituted derivatives \\ 
\hspace*{0.6cm}Mecoprop & 7085-19-0 & $\checkmark$ & 2B & Benzenoids: Benzene and substituted derivatives \\ 
\hspace*{0.4cm}\textit{\underline{Phenylurea}} &  &  &  &  \\ 
\hspace*{0.6cm}Diuron & 330-54-1 & $\checkmark$ & – & Benzenoids: Benzene and substituted derivatives \\ 
\hspace*{0.6cm}Linuron & 330-55-2 & $\checkmark$ & – & Benzenoids: Benzene and substituted derivatives \\ 
\hspace*{0.6cm}Monuron & 150-68-5 & – & 3 & Benzenoids: Benzene and substituted derivatives \\ 
\hspace*{0.6cm}Siduron & 1982-49-6 & – & – & Benzenoids: Benzene and substituted derivatives \\ 
\hspace*{0.6cm}Tebuthiuron & 34014-18-1 & – & – & Organoheterocyclic compounds: Azoles \\ 
\hspace*{0.4cm}\textit{\underline{Sulfonylurea}} &  &  &  &  \\ 
\hspace*{0.6cm}Bensulfuron-methyl & 83055-99-6 & – & – & Benzenoids: Benzene and substituted derivatives \\ 
\hspace*{0.6cm}Halosulfuron-methyl & 100784-20-1 & – & – & Organoheterocyclic compounds: Azoles \\ 
\hspace*{0.6cm}Metsulfuron-methyl & 74223-64-6 & – & – & Benzenoids: Benzene and substituted derivatives \\ 
\hspace*{0.6cm}Nicosulfuron & 111991-09-4 & – & – & Organoheterocyclic compounds: Pyridines and derivatives \\ 
\hspace*{0.6cm}Oxasulfuron & 144651-06-9 & $\checkmark$ & – & Organic nitrogen compounds: Organonitrogen compounds \\ 
\hspace*{0.6cm}Primisulfuron-methyl & 86209-51-0 & – & – & Organic nitrogen compounds: Organonitrogen compounds \\ 
\hspace*{0.6cm}Prosulfuron & 94125-34-5 & $\checkmark$ & – & Benzenoids: Benzene and substituted derivatives \\ 
\hspace*{0.6cm}Rimsulfuron & 122931-48-0 & $\checkmark$ & – & Organoheterocyclic compounds: Pyridines and derivatives \\ 
\hspace*{0.6cm}Thifensulfuron-methyl & 79277-27-3 & $\checkmark$ & – & Organoheterocyclic compounds: Thiophenes \\ 
\hspace*{0.6cm}Triasulfuron & 82097-50-5 & – & – & Benzenoids: Benzene and substituted derivatives \\ 
\hspace*{0.4cm}\textit{\underline{Thiocarbamate}} &  &  &  &  \\ 
\hspace*{0.6cm}Butylate & 2008-41-5 & – & – & Organosulfur compounds: Thiocarbonyl compounds \\ 
\hspace*{0.6cm}Cycloate & 1134-23-2 & – & – & Organosulfur compounds: Thiocarbonyl compounds \\ 
\hspace*{0.6cm}Diallate-cis & 2303-16-4 & – & 3 & Organosulfur compounds: Thiocarbonyl compounds \\ 
\hspace*{0.6cm}Molinate & 2212-67-1 & $\checkmark$ & – & Organoheterocyclic compounds: Azepanes \\ 
\hspace*{0.6cm}Pebulate & 1114-71-2 & – & – & Organosulfur compounds: Thiocarbonyl compounds \\ 
\hspace*{0.6cm}S-Ethyl dipropylthiocarbamate & 759-94-4 & – & – & Organosulfur compounds: Thiocarbonyl compounds \\ 
\hspace*{0.6cm}Triallate & 2303-17-5 & – & – & Organosulfur compounds: Thiocarbonyl compounds \\ 
\hspace*{0.6cm}Vernolate & 1929-77-7 & – & – & Organosulfur compounds: Thiocarbonyl compounds \\ 
\hspace*{0.4cm}\textit{\underline{Triazine}} &  &  &  &  \\ 
\hspace*{0.6cm}Atraton & 1610-17-9 & – & – & Organoheterocyclic compounds: Triazines \\ 
\hspace*{0.6cm}Atrazine & 1912-24-9 & $\checkmark$ & 3 & Organoheterocyclic compounds: Triazines \\ 
\hspace*{0.6cm}Hexazinone & 51235-04-2 & – & – & Organic nitrogen compounds: Organonitrogen compounds \\ 
\hspace*{0.6cm}Metribuzin & 21087-64-9 & $\checkmark$ & – & Organosulfur compounds: Thioethers \\ 
\hspace*{0.6cm}Propazine & 139-40-2 & $\checkmark$ & – & Organoheterocyclic compounds: Triazines \\ 
\hspace*{0.6cm}Simetryn & 1014-70-6 & – & – & Organoheterocyclic compounds: Triazines \\ 
\hspace*{0.6cm}Terbuthylazine & 5915-41-3 & $\checkmark$ & – & Organoheterocyclic compounds: Triazines \\ 
\hspace*{0.6cm}Terbutryn & 886-50-0 & – & – & Organoheterocyclic compounds: Triazines \\ 
\hspace*{0.4cm}\textit{\underline{Uracil}} &  &  &  &  \\ 
\hspace*{0.6cm}Bromacil & 314-40-9 & – & – & Organoheterocyclic compounds: Diazines \\ 
\hspace*{0.6cm}Lenacil & 2164-08-1 & $\checkmark$ & – & Organoheterocyclic compounds: Diazines \\ 
\hspace*{0.6cm}Terbacil & 5902-51-2 & – & – & Organoheterocyclic compounds: Diazines \\ 
\hspace*{0.4cm}\textit{\underline{Other}} &  &  &  &  \\ 
\hspace*{0.6cm}Bentazon & 25057-89-0 & – & – & Benzenoids: NA \\ 
\hspace*{0.6cm}Clomazone & 81777-89-1 & $\checkmark$ & – & Benzenoids: Benzene and substituted derivatives \\ 
\hspace*{0.6cm}Dicamba & 1918-00-9 & $\checkmark$ & – & Benzenoids: Benzene and substituted derivatives \\ 
\hspace*{0.6cm}Fluridone & 59756-60-4 & – & – & Organoheterocyclic compounds: Pyridines and derivatives \\ 
\hspace*{0.6cm}Glyphosate & 1071-83-6 & $\checkmark$ & 2A & Organic acids and derivatives: Carboxylic acids and derivatives \\ 
\hspace*{0.6cm}Oxadiazon & 19666-30-9 & $\checkmark$ & – & Benzenoids: Benzene and substituted derivatives \\ 
 &  &  &  &  \\ 
 &  &  &  &  \\ 
\hspace*{0.2cm}\textbf{Insecticides and Pesticides} &  &  &  &  \\ 
\hspace*{0.4cm}\textit{\underline{Carbamate}} &  &  &  &  \\ 
\hspace*{0.6cm}3-Hydroxycarbofuran & 16655-82-6 & – & – & Organoheterocyclic compounds: Coumarans \\ 
\hspace*{0.6cm}Aldicarb & 116-06-3 & – & 3 & Organic acids and derivatives: Carboximidic acids and derivatives \\ 
\hspace*{0.6cm}Aldicarb sulfone & 1646-88-4 & – & – & Organosulfur compounds: Sulfonyls \\ 
\hspace*{0.6cm}Aldicarb sulfoxide & 1646-87-3 & – & – & Organic nitrogen compounds: Organonitrogen compounds \\ 
\hspace*{0.6cm}Carbaryl & 63-25-2 & $\checkmark$ & 3 & Benzenoids: Naphthalenes \\ 
\hspace*{0.6cm}Carbofuran & 1563-66-2 & $\checkmark$ & – & Organoheterocyclic compounds: Coumarans \\ 
\hspace*{0.6cm}Methomyl & 16752-77-5 & – & – & Organic acids and derivatives: Organic carbonic acids and derivatives \\ 
\hspace*{0.6cm}Oxamyl & 23135-22-0 & $\checkmark$ & – & Organic acids and derivatives: Carboxylic acids and derivatives \\ 
\hspace*{0.6cm}propoxur & 114-26-1 & – & – & Benzenoids: Phenol ethers \\ 
\hspace*{0.4cm}\textit{\underline{Insect Repellents}} &  &  &  &  \\ 
\hspace*{0.6cm}N,N-Diethyl-meta-toluamide & 134-62-3 & – & – & Benzenoids: Benzene and substituted derivatives \\ 
\hspace*{0.4cm}\textit{\underline{Organochlorine}} &  &  &  &  \\ 
\hspace*{0.6cm}2,4'-DDT & 789-02-6 & $\checkmark$ & 2B & Benzenoids: Benzene and substituted derivatives \\ 
\hspace*{0.6cm}2,4'-Methoxychlor & 30667-99-3 & – & – & Benzenoids: Benzene and substituted derivatives \\ 
\hspace*{0.6cm}4,4'-DDD & 72-54-8 & $\checkmark$ & – & Benzenoids: Benzene and substituted derivatives \\ 
\hspace*{0.6cm}4,4'-DDE & 72-55-9 & $\checkmark$ & 2B & Benzenoids: Benzene and substituted derivatives \\ 
\hspace*{0.6cm}4,4'-DDT & 50-29-3 & $\checkmark$ & 2A & Benzenoids: Benzene and substituted derivatives \\ 
\hspace*{0.6cm}4,4'-Methoxychlor olefin & 2132-70-9 & – & – & Benzenoids: Benzene and substituted derivatives \\ 
\hspace*{0.6cm}B8-1413 & 142534-71-2 & $\checkmark$ & – & Lipids and lipid-like molecules: Prenol lipids \\ 
\hspace*{0.6cm}B9-1025 & 154159-06-5 & – & – & Lipids and lipid-like molecules: Prenol lipids \\ 
\hspace*{0.6cm}Butylphenoxyisopropyl chloroethyl sulfite & 140-57-8 & – & 2B & Benzenoids: Benzene and substituted derivatives \\ 
\hspace*{0.6cm}Chlorfenson & 80-33-1 & – & – & Benzenoids: Benzene and substituted derivatives \\ 
\hspace*{0.6cm}Endosulfan I & 959-98-8 & $\checkmark$ & – & Organic oxygen compounds: Organic oxoanionic compounds \\ 
\hspace*{0.6cm}Methoxychlor & 72-43-5 & $\checkmark$ & 3 & Benzenoids: Benzene and substituted derivatives \\ 
\hspace*{0.6cm}alpha-benzenehexachloride & 319-84-6 & – & 2B & Organohalogen compounds: Alkyl halides \\ 
\hspace*{0.6cm}beta-benzenehexachloride & 319-85-7 & $\checkmark$ & 2B & Organohalogen compounds: Alkyl halides \\ 
\hspace*{0.6cm}cis-Nonachlor & 5103-73-1 & – & – & Organohalogen compounds: Vinyl halides \\ 
\hspace*{0.6cm}ethyl-DDD & 72-56-0 & – & – & Lipids and lipid-like molecules: Prenol lipids \\ 
\hspace*{0.6cm}gamma-benzenehexachloride & 58-89-9 & $\checkmark$ & 1 & Organohalogen compounds: Alkyl halides \\ 
\hspace*{0.6cm}o,p'-DDD & 53-19-0 & $\checkmark$ & – & Benzenoids: Benzene and substituted derivatives \\ 
\hspace*{0.4cm}\textit{\underline{Organophosphate}} &  &  &  &  \\ 
\hspace*{0.6cm}Azinphos ethyl & 2642-71-9 & – & – & Organoheterocyclic compounds: Benzo-1,2,3-triazines \\ 
\hspace*{0.6cm}Bromfenvinphos (E) & 33399-00-7 & – & – & Benzenoids: Benzene and substituted derivatives \\ 
\hspace*{0.6cm}Chlorpyrifos & 2921-88-2 & $\checkmark$ & – & Organic acids and derivatives: Organic thiophosphoric acids and derivatives \\ 
\hspace*{0.6cm}Chlorthiophos & 21923-23-9 & – & – & Organic acids and derivatives: Organic thiophosphoric acids and derivatives \\ 
\hspace*{0.6cm}Diazinon & 333-41-5 & $\checkmark$ & 2A & Organic acids and derivatives: Organic thiophosphoric acids and derivatives \\ 
\hspace*{0.6cm}Diethyl phosphate & 598-02-7 & – & – & Organic acids and derivatives: Organic phosphoric acids and derivatives \\ 
\hspace*{0.6cm}Disulfoton & 298-04-4 & – & – & Organic acids and derivatives: Organic dithiophosphoric acids and derivatives \\ 
\hspace*{0.6cm}Ethyl parathion & 56-38-2 & $\checkmark$ & 2B & Organic acids and derivatives: Organic thiophosphoric acids and derivatives \\ 
\hspace*{0.6cm}Fenamiphos & 22224-92-6 & – & – & Benzenoids: Benzene and substituted derivatives \\ 
\hspace*{0.6cm}Fenthion & 55-38-9 & $\checkmark$ & – & Organic acids and derivatives: Organic thiophosphoric acids and derivatives \\ 
\hspace*{0.6cm}Fonofos & 944-22-9 & – & – & Benzenoids: Benzene and substituted derivatives \\ 
\hspace*{0.6cm}Iodofenphos & 18181-70-9 & – & – & Organic acids and derivatives: Organic thiophosphoric acids and derivatives \\ 
\hspace*{0.6cm}Malathion & 121-75-5 & $\checkmark$ & 2A & Lipids and lipid-like molecules: Fatty Acyls \\ 
\hspace*{0.6cm}Methyl parathion & 298-00-0 & $\checkmark$ & 3 & Organic acids and derivatives: Organic thiophosphoric acids and derivatives \\ 
\hspace*{0.6cm}Mevinphos & 7786-34-7 & – & – & Lipids and lipid-like molecules: Fatty Acyls \\ 
\hspace*{0.6cm}O,O-Dimethyl Thiophosphate & 1112-38-5 & – & – & Organic acids and derivatives: Organic thiophosphoric acids and derivatives \\ 
\hspace*{0.6cm}Phorate & 298-02-2 & – & – & Organic acids and derivatives: Organic dithiophosphoric acids and derivatives \\ 
\hspace*{0.6cm}Phosmet & 732-11-6 & – & – & Organoheterocyclic compounds: Isoindoles and derivatives \\ 
\hspace*{0.6cm}Pirimiphos ethyl & 23505-41-1 & – & – & Organic acids and derivatives: Organic thiophosphoric acids and derivatives \\ 
\hspace*{0.6cm}Pirimiphos methyl & 29232-93-7 & $\checkmark$ & – & Organic acids and derivatives: Organic thiophosphoric acids and derivatives \\ 
\hspace*{0.6cm}Pyraclofos & 89784-60-1 & – & – & Organoheterocyclic compounds: Azoles \\ 
\hspace*{0.6cm}Pyridaphenthion & 119-12-0 & – & – & Organic acids and derivatives: Organic thiophosphoric acids and derivatives \\ 
\hspace*{0.6cm}Quinalphos & 13593-03-8 & $\checkmark$ & – & Organoheterocyclic compounds: Diazanaphthalenes \\ 
\hspace*{0.6cm}Resmethrin & 10453-86-8 & $\checkmark$ & – & Lipids and lipid-like molecules: Prenol lipids \\ 
\hspace*{0.6cm}Sulprofos & 35400-43-2 & – & – & Benzenoids: Benzene and substituted derivatives \\ 
\hspace*{0.6cm}Tetraethyl ethylenediphosphonate & 995-32-4 & – & – & Organic acids and derivatives: Organic phosphonic acids and derivatives \\ 
\hspace*{0.6cm}Triazophos & 24017-47-8 & – & – & Organoheterocyclic compounds: Azoles \\ 
\hspace*{0.4cm}\textit{\underline{Pyrethroid}} &  &  &  &  \\ 
\hspace*{0.6cm}3-Phenoxybenzoic acid & 3739-38-6 & – & – & Benzenoids: Benzene and substituted derivatives \\ 
\hspace*{0.6cm}4-Fluro-3-phenoxy-benzoic acid & 77279-89-1 & – & – & Benzenoids: Benzene and substituted derivatives \\ 
\hspace*{0.6cm}Acrinathrin & 101007-06-1 & $\checkmark$ & – & Lipids and lipid-like molecules: Fatty Acyls \\ 
\hspace*{0.6cm}Bifenthrin & 82657-04-3 & $\checkmark$ & – & Benzenoids: Benzene and substituted derivatives \\ 
\hspace*{0.6cm}Bioallethrin & 584-79-2 & – & – & Lipids and lipid-like molecules: Fatty Acyls \\ 
\hspace*{0.6cm}Cyfluthrin & 68359-37-5 & $\checkmark$ & – & Lipids and lipid-like molecules: Fatty Acyls \\ 
\hspace*{0.6cm}Cypermethrin & 52315-07-8 & $\checkmark$ & – & Lipids and lipid-like molecules: Fatty Acyls \\ 
\hspace*{0.6cm}Deltamethrin & 52918-63-5 & $\checkmark$ & 3 & Lipids and lipid-like molecules: Fatty Acyls \\ 
\hspace*{0.6cm}Etofenprox & 80844-07-1 & $\checkmark$ & – & Benzenoids: Benzene and substituted derivatives \\ 
\hspace*{0.6cm}Fenpropathrin & 39515-41-8 & – & – & Benzenoids: Benzene and substituted derivatives \\ 
\hspace*{0.6cm}Fenvalerate & 51630-58-1 & $\checkmark$ & 3 & Lipids and lipid-like molecules: Fatty Acyls \\ 
\hspace*{0.6cm}Flucythrinate & 70124-77-5 & – & – & Lipids and lipid-like molecules: Fatty Acyls \\ 
\hspace*{0.6cm}Phenothrin, trans & 26002-80-2 & $\checkmark$ & – & Lipids and lipid-like molecules: Fatty Acyls \\ 
\hspace*{0.6cm}Tefluthrin & 79538-32-2 & $\checkmark$ & – & Benzenoids: Benzene and substituted derivatives \\ 
\hspace*{0.6cm}Tetramethrin & 7696-12-0 & $\checkmark$ & – & Organoheterocyclic compounds: Isoindoles and derivatives \\ 
\hspace*{0.6cm}Transfluthrin & 118712-89-3 & – & – & Benzenoids: Benzene and substituted derivatives \\ 
\hspace*{0.6cm}cis-3-(2,2-Dichlorovinyl)-2,2-dimethylcylo & 59042-49-8 & – & – & Organic acids and derivatives: Carboxylic acids and derivatives \\ 
\hspace*{0.6cm}cis-Permethrin & 61949-76-6 & $\checkmark$ & 3 & Lipids and lipid-like molecules: Fatty Acyls \\ 
\hspace*{0.6cm}lambda-Cyhalothrin & 91465-08-6 & $\checkmark$ & – & Lipids and lipid-like molecules: Fatty Acyls \\ 
\hspace*{0.6cm}tau-Fluvalinate & 102851-06-9 & $\checkmark$ & – & Benzenoids: Benzene and substituted derivatives \\ 
\hspace*{0.6cm}trans-3-(2,2-Dichlorovinyl)-2,2-dimethylcylo & 59042-50-1 & – & – & Organic acids and derivatives: Carboxylic acids and derivatives \\ 
\hspace*{0.6cm}trans-Permethrin & 61949-77-7 & – & 3 & Lipids and lipid-like molecules: Fatty Acyls \\ 
\hspace*{0.4cm}\textit{\underline{Pyrrole}} &  &  &  &  \\ 
\hspace*{0.6cm}Chlorfenapyr & 122453-73-0 & – & – & Organoheterocyclic compounds: Pyrroles \\ 
\hspace*{0.4cm}\textit{\underline{Synergists}} &  &  &  &  \\ 
\hspace*{0.6cm}Octyl bicycloheptenedicarboximide & 113-48-4 & – & – & Organoheterocyclic compounds: Isoindoles and derivatives \\ 
\hspace*{0.6cm}Piperonyl butoxide & 51-03-6 & – & 3 & Organoheterocyclic compounds: Benzodioxoles \\ 
 &  &  &  &  \\ 
\hspace*{0.2cm}\textbf{Plant Growth Regulators} &  &  &  &  \\ 
\hspace*{0.6cm}2,3,4,5-Tetrabromobenzoic acid & 27581-13-1 & – & – & Benzenoids: Benzene and substituted derivatives \\ 
\hspace*{0.6cm}Chlorpropham & 101-21-3 & $\checkmark$ & 3 & Benzenoids: Benzene and substituted derivatives \\ 
\hspace*{0.6cm}Paclobutrazol & 76738-62-0 & $\checkmark$ & – & Benzenoids: Benzene and substituted derivatives \\ 
 &  &  &  &  \\ 
 &  &  &  &  \\ 
\newpage
{\fontsize{10pt}{12pt}\selectfont\textbf{OTHER CHEMICALS}} &  &  &  &  \\ 
\hspace*{0.2cm}\textbf{Antioxidants} &  &  &  &  \\ 
\hspace*{0.6cm}Protocatechuic acidᵃ & 99-50-3 & $\checkmark$ & – & Benzenoids: Benzene and substituted derivatives \\ 
 &  &  &  &  \\ 
\hspace*{0.2cm}\textbf{Emollients} &  &  &  &  \\ 
\hspace*{0.6cm}Ethyl myristate & 124-06-1 & – & – & Lipids and lipid-like molecules: Fatty Acyls \\ 
\hspace*{0.6cm}Ethyl oleate & 111-62-6 & – & – & Lipids and lipid-like molecules: Fatty Acyls \\ 
\hspace*{0.6cm}Ethyl palmitate & 628-97-7 & – & – & Lipids and lipid-like molecules: Fatty Acyls \\ 
 &  &  &  &  \\ 
\hspace*{0.2cm}\textbf{Flavoring or Fragrance Agents} &  &  &  &  \\ 
\hspace*{0.6cm}(-)-Menthol & 2216-51-5 & – & – & Lipids and lipid-like molecules: Prenol lipids \\ 
\hspace*{0.6cm}1,8-Cineole & 470-82-6 & – & – & Organoheterocyclic compounds: Oxanes \\ 
\hspace*{0.6cm}4-Hydroxy-2,5-dimethyl-3(2H)-furanone & 3658-77-3 & – & – & Organoheterocyclic compounds: Dihydrofurans \\ 
\hspace*{0.6cm}Acetophenone & 98-86-2 & – & – & Organic oxygen compounds: Organooxygen compounds \\ 
\hspace*{0.6cm}Benzyl salicylate & 118-58-1 & $\checkmark$ & – & Benzenoids: Benzene and substituted derivatives \\ 
\hspace*{0.6cm}Cinnamyl acetate & 103-54-8 & – & – & Benzenoids: Benzene and substituted derivatives \\ 
\hspace*{0.6cm}Cinnamyl alcohol & 104-54-1 & $\checkmark$ & – & Phenylpropanoids and polyketides: Cinnamyl alcohols \\ 
\hspace*{0.6cm}Ethyl butyrate & 105-54-4 & – & – & Lipids and lipid-like molecules: Fatty Acyls \\ 
\hspace*{0.6cm}Ethyl caprateᵃ & 110-38-3 & – & – & Lipids and lipid-like molecules: Fatty Acyls \\ 
\hspace*{0.6cm}Ethyl caproate & 123-66-0 & – & – & Lipids and lipid-like molecules: Fatty Acyls \\ 
\hspace*{0.6cm}Ethyl caprylate & 106-32-1 & – & – & Lipids and lipid-like molecules: Fatty Acyls \\ 
\hspace*{0.6cm}Ethyl heptanoate & 106-30-9 & – & – & Lipids and lipid-like molecules: Fatty Acyls \\ 
\hspace*{0.6cm}Ethyl laurate & 106-33-2 & – & – & Lipids and lipid-like molecules: Fatty Acyls \\ 
\hspace*{0.6cm}Ethyl maltol & 225-582-5 & – & – & Organoheterocyclic compounds: Pyrans \\ 
\hspace*{0.6cm}Ethyl pelargonateᵃ & 123-29-5 & – & – & Lipids and lipid-like molecules: Fatty Acyls \\ 
\hspace*{0.6cm}Ethyl propionate & 105-37-3 & – & – & Organic acids and derivatives: Carboxylic acids and derivatives \\ 
\hspace*{0.6cm}Ethyl stearateᵃ & 111-61-5 & – & – & Lipids and lipid-like molecules: Fatty Acyls \\ 
\hspace*{0.6cm}Ethyl undecanoateᵃ & 627-90-7 & – & – & Lipids and lipid-like molecules: Fatty Acyls \\ 
\hspace*{0.6cm}Ethyl valerate & 539-82-2 & – & – & Lipids and lipid-like molecules: Fatty Acyls \\ 
\hspace*{0.6cm}Ethyl vanillin & 121-32-4 & – & – & Organic oxygen compounds: Organooxygen compounds \\ 
\hspace*{0.6cm}Geraniol & 106-24-1 & – & – & Lipids and lipid-like molecules: Prenol lipids \\ 
\hspace*{0.6cm}Hexanoic acid & 142-62-1 & – & – & Lipids and lipid-like molecules: Fatty Acyls \\ 
\hspace*{0.6cm}Hydroxycitronellal & 107-75-5 & – & – & Organic oxygen compounds: Organooxygen compounds \\ 
\hspace*{0.6cm}Isosafrole & 120-58-1 & – & 3 & Organoheterocyclic compounds: Benzodioxoles \\ 
\hspace*{0.6cm}Maltol & 118-71-8 & – & – & Organoheterocyclic compounds: Pyrans \\ 
\hspace*{0.6cm}Menthone & 14073-97-3 & – & – & Lipids and lipid-like molecules: Prenol lipids \\ 
\hspace*{0.6cm}Safrole & 94-59-7 & – & 2B & Organoheterocyclic compounds: Benzodioxoles \\ 
\hspace*{0.6cm}Thymol & 89-83-8 & $\checkmark$ & – & Lipids and lipid-like molecules: Prenol lipids \\ 
\hspace*{0.6cm}Vanillin & 121-33-5 & – & – & Benzenoids: Phenols \\ 
\hspace*{0.6cm}d-Limonene & 5989-27-5 & – & 3 & Lipids and lipid-like molecules: Prenol lipids \\ 
\hspace*{0.6cm}trans-Cinnamaldehyde & 14371-10-9 & $\checkmark$ & – & Phenylpropanoids and polyketides: Cinnamaldehydes \\ 
 &  &  &  &  \\ 
\hspace*{0.2cm}\textbf{Humectants} &  &  &  &  \\ 
\hspace*{0.6cm}Glycerinᵃ & 56-81-5 & – & – & Organic oxygen compounds: Organooxygen compounds \\ 
 &  &  &  &  \\ 
\hspace*{0.2cm}\textbf{Organic UV Filters} &  &  &  &  \\ 
\hspace*{0.6cm}Butyl methoxydibenzoylmethane & 70356-09-1 & – & – & Phenylpropanoids and polyketides: Linear 1,3-diarylpropanoids \\ 
\hspace*{0.6cm}Homosalate & 118-56-9 & $\checkmark$ & – & Benzenoids: Benzene and substituted derivatives \\ 
\hspace*{0.6cm}Octyl-dimethyl-p-aminobenzoic acid & 21245-02-3 & $\checkmark$ & – & Benzenoids: Benzene and substituted derivatives \\ 
\hspace*{0.6cm}Oxybenzone & 131-57-7 & $\checkmark$ & – & Benzenoids: Benzene and substituted derivatives \\ 
\hspace*{0.6cm}octyl-Methoxycinnamate & 5466-77-3 & $\checkmark$ & – & Phenylpropanoids and polyketides: Cinnamic acids and derivatives \\ 
 &  &  &  &  \\ 
 &  &  &  &  \\ 
\hspace*{0.2cm}\textbf{Pharmacologic Agents} &  &  &  &  \\ 
\hspace*{0.6cm}(±)-Nicotine & 22083-74-5 & – & – & Organoheterocyclic compounds: Pyridines and derivatives \\ 
\hspace*{0.6cm}Caffeine & 58-08-2 & $\checkmark$ & 3 & Organoheterocyclic compounds: Imidazopyrimidines \\ 
\hspace*{0.6cm}Methapyrilene & 91-80-5 & – & – & Organic nitrogen compounds: Organonitrogen compounds \\ 
\hspace*{0.6cm}Phenacetin & 62-44-2 & – & 1 & Benzenoids: Benzene and substituted derivatives \\ 
 &  &  &  &  \\ 
\hspace*{0.2cm}\textbf{Preservatives} &  &  &  &  \\ 
\hspace*{0.4cm}\textit{\underline{Parabens}} &  &  &  &  \\ 
\hspace*{0.6cm}Benzylparaben & 94-18-8 & $\checkmark$ & – & Benzenoids: Benzene and substituted derivatives \\ 
\hspace*{0.6cm}Butylparaben & 94-26-8 & $\checkmark$ & – & Benzenoids: Benzene and substituted derivatives \\ 
\hspace*{0.6cm}Ethylparaben & 120-47-8 & $\checkmark$ & – & Benzenoids: Benzene and substituted derivatives \\ 
\hspace*{0.6cm}Heptylparaben & 1085-12-7 & $\checkmark$ & – & Benzenoids: Benzene and substituted derivatives \\ 
\hspace*{0.6cm}Isobutylparaben & 4247-02-3 & $\checkmark$ & – & Benzenoids: Benzene and substituted derivatives \\ 
\hspace*{0.6cm}Isopropylparaben & 4191-73-5 & $\checkmark$ & – & Benzenoids: Benzene and substituted derivatives \\ 
\hspace*{0.6cm}Methylparaben & 99-76-3 & $\checkmark$ & – & Benzenoids: Benzene and substituted derivatives \\ 
\hspace*{0.6cm}Propylparaben & 94-13-3 & $\checkmark$ & – & Benzenoids: Benzene and substituted derivatives \\ 
\hspace*{0.4cm}\textit{\underline{Other}} &  &  &  &  \\ 
\hspace*{0.6cm}4-Chloro-3-methylphenol & 59-50-7 & $\checkmark$ & – & Benzenoids: Phenols \\ 
\hspace*{0.6cm}4-Hydroxybenzoic acid & 99-96-7 & $\checkmark$ & – & Benzenoids: Benzene and substituted derivatives \\ 
\hspace*{0.6cm}Benzoic acid & 65-85-0 & – & – & Benzenoids: Benzene and substituted derivatives \\ 
\hspace*{0.6cm}Benzyl alcohol & 100-51-6 & – & – & Benzenoids: Benzene and substituted derivatives \\ 
\hspace*{0.6cm}Potassium sorbate & 24634-61-5 & – & – & Lipids and lipid-like molecules: Fatty Acyls \\ 
\hspace*{0.6cm}Triclosan & 3380-34-5 & $\checkmark$ & – & Benzenoids: Benzene and substituted derivatives \\ 
 &  &  &  &  \\ 
 &  &  &  &  \\ 
\newpage
{\fontsize{10pt}{12pt}\selectfont\textbf{POLLUTANTS AND INDUSTRIAL CHEMICALS}} &  &  &  &  \\ 
\hspace*{0.2cm}\textbf{Carcinogenic Research Chemicals} &  &  &  &  \\ 
\hspace*{0.6cm}2-Acetylaminofluorene & 53-96-3 & – & – & -: Fluorenes \\ 
\hspace*{0.6cm}4-Nitroquinoline-1-oxide & 56-57-5 & – & – & -: Quinolines and derivatives \\ 
\hspace*{0.6cm}Methyl methanesulfonate & 66-27-3 & – & 2A & -: Organic sulfonic acids and derivatives \\ 
\hspace*{0.6cm}N-Nitrosodi-n-butylamine & 924-16-3 & – & 2B & -: Organonitrogen compounds \\ 
\hspace*{0.6cm}N-Nitrosoethylmethylamine & 10595-95-6 & – & 2B & -: Organonitrogen compounds \\ 
\hspace*{0.6cm}N-Nitrosopyrrolidine & 930-55-2 & – & 2B & -: Pyrrolidines \\ 
 &  &  &  &  \\ 
\hspace*{0.2cm}\textbf{Chemical Synthesis Intermediates} &  &  &  &  \\ 
\hspace*{0.6cm}1,2,3-Trichloro-4-nitrobenzene & 17700-09-3 & – & – & Benzenoids: Benzene and substituted derivatives \\ 
\hspace*{0.6cm}1,3-Dichlorobenzene & 541-73-1 & $\checkmark$ & 3 & Benzenoids: Benzene and substituted derivatives \\ 
\hspace*{0.6cm}1,4-Naphthoquinone & 130-15-4 & – & – & Benzenoids: Naphthalenes \\ 
\hspace*{0.6cm}2,2-Dichloropropane & 594-20-7 & – & – & Organohalogen compounds: Organochlorides \\ 
\hspace*{0.6cm}2,3-Dichloronitrobenzene & 3209-22-1 & – & – & Benzenoids: Benzene and substituted derivatives \\ 
\hspace*{0.6cm}2,3-Dichlorophenyl-4'-nitrophenyl ether & 82239-20-1 & – & – & Benzenoids: Benzene and substituted derivatives \\ 
\hspace*{0.6cm}2,4,6-Tribromophenol & 118-79-6 & $\checkmark$ & – & Benzenoids: Phenols \\ 
\hspace*{0.6cm}2,4-Dichlorophenol & 120-83-2 & $\checkmark$ & 2B & Benzenoids: Benzene and substituted derivatives \\ 
\hspace*{0.6cm}2,4-Dimethylphenol & 105-67-9 & – & – & Benzenoids: Benzene and substituted derivatives \\ 
\hspace*{0.6cm}2,4-Dinitrophenol & 51-28-5 & – & – & Benzenoids: Phenols \\ 
\hspace*{0.6cm}2,6-Dichlorophenyl-4'-nitrophenyl ether & 2093-28-9 & – & – & Benzenoids: Benzene and substituted derivatives \\ 
\hspace*{0.6cm}2-Bromo-4-(2,4,4-trimethylpentan-2-yl)phenol & 57835-35-5 & – & – & Benzenoids: Benzene and substituted derivatives \\ 
\hspace*{0.6cm}2-Chloronaphthalene & 91-58-7 & – & – & Benzenoids: Naphthalenes \\ 
\hspace*{0.6cm}2-Isopropyl-6-methyl-4-pyrimidinol & 2814-20-2 & – & – & Organoheterocyclic compounds: Diazines \\ 
\hspace*{0.6cm}2-Nitrophenol & 88-75-5 & – & – & Benzenoids: Phenols \\ 
\hspace*{0.6cm}3,5-Dichloronitrobenzene & 618-62-2 & – & – & Benzenoids: Benzene and substituted derivatives \\ 
\hspace*{0.6cm}3,5-Dichlorophenyl-4'-nitrophenyl ether & 21105-77-1 & – & – & Benzenoids: Benzene and substituted derivatives \\ 
\hspace*{0.6cm}4,4'-Diaminodiphenylmethane & 101-77-9 & $\checkmark$ & 2B & Benzenoids: Benzene and substituted derivatives \\ 
\hspace*{0.6cm}4,4'-Methylenebis(2-chloroaniline) & 101-14-4 & – & 1 & Benzenoids: Benzene and substituted derivatives \\ 
\hspace*{0.6cm}4,4'-Oxydianiline & 101-80-4 & – & 2B & Benzenoids: Benzene and substituted derivatives \\ 
\hspace*{0.6cm}4-Chloro-o-toluidine & 95-69-2 & – & 2A & Benzenoids: Benzene and substituted derivatives \\ 
\hspace*{0.6cm}4-Chlorophenyl phenyl ether & 7005-72-3 & – & – & Benzenoids: Benzene and substituted derivatives \\ 
\hspace*{0.6cm}4-Nitrophenyl phenyl ether & 620-88-2 & – & – & Benzenoids: Benzene and substituted derivatives \\ 
\hspace*{0.6cm}Carbofuran phenol & 1563-38-8 & – & – & Organoheterocyclic compounds: Coumarans \\ 
\hspace*{0.6cm}Dibenzofuran & 132-64-9 & – & – & Organoheterocyclic compounds: Benzofurans \\ 
\hspace*{0.6cm}Dimethyl phosphate & 813-78-5 & – & – & Organic acids and derivatives: Organic phosphoric acids and derivatives \\ 
\hspace*{0.6cm}Diphenylamine & 122-39-4 & – & 2B & Benzenoids: Benzene and substituted derivatives \\ 
\hspace*{0.6cm}Ethyl methacrylate & 97-63-2 & – & – & Organic acids and derivatives: Carboxylic acids and derivatives \\ 
\hspace*{0.6cm}Ethylene thiourea & 96-45-7 & $\checkmark$ & 3 & Organoheterocyclic compounds: Azolidines \\ 
\hspace*{0.6cm}Hexachloropropene & 1888-71-7 & – & – & Organohalogen compounds: Vinyl halides \\ 
\hspace*{0.6cm}Isopropylbenzene & 98-82-8 & – & 2B & Benzenoids: Benzene and substituted derivatives \\ 
\hspace*{0.6cm}Methyl methacrylate & 80-62-6 & $\checkmark$ & 3 & Organic acids and derivatives: Carboxylic acids and derivatives \\ 
\hspace*{0.6cm}N-(2,4-Dimethylphenyl)formamide & 60397-77-5 & – & – & Benzenoids: Benzene and substituted derivatives \\ 
\hspace*{0.6cm}N-Nitrosodiethylamine & 55-18-5 & – & 2A & Organic nitrogen compounds: Organonitrogen compounds \\ 
\hspace*{0.6cm}N-ethylperfluoro-1-octanesulfonamidoacetic acid (linear) & 2991-50-6 & – & – & Organohalogen compounds: Alkyl halides \\ 
\hspace*{0.6cm}Nitrobenzene & 98-95-3 & $\checkmark$ & 2B & Benzenoids: Benzene and substituted derivatives \\ 
\hspace*{0.6cm}O,O-Diethyl Phosphonate & 762-04-9 & – & – & Organic oxygen compounds: Organooxygen compounds \\ 
\hspace*{0.6cm}Phenol, 2-octyl- & 949-13-3 & $\checkmark$ & – & Benzenoids: Phenols \\ 
\hspace*{0.6cm}Phenolᵃ & 108-95-2 & – & 3 & Benzenoids: Phenols \\ 
\hspace*{0.6cm}Propylene thiourea & 2122-19-2 & – & – & Organoheterocyclic compounds: Azolidines \\ 
\hspace*{0.6cm}Styrene & 100-42-5 & $\checkmark$ & 2A & Benzenoids: Benzene and substituted derivatives \\ 
\hspace*{0.6cm}Tetrahydrophthalimide & 85-40-5 & – & – & Organoheterocyclic compounds: Isoindoles and derivatives \\ 
\hspace*{0.6cm}Triisopropyl Phosphate & 513-02-0 & – & – & Organic acids and derivatives: Organic phosphoric acids and derivatives \\ 
\hspace*{0.6cm}Trimethyl Phosphate & 512-56-1 & – & – & Organic acids and derivatives: Organic phosphoric acids and derivatives \\ 
\hspace*{0.6cm}bis(2-Chloroethyl)ether & 111-44-4 & – & 3 & Organic oxygen compounds: Organooxygen compounds \\ 
\hspace*{0.6cm}m-Cresol & 108-39-4 & – & – & Benzenoids: Phenols \\ 
\hspace*{0.6cm}o-Cresol & 95-48-7 & – & – & Benzenoids: Phenols \\ 
\hspace*{0.6cm}p-Cresol & 106-44-5 & $\checkmark$ & – & Benzenoids: Phenols \\ 
\hspace*{0.6cm}p-Nitrophenol & 100-02-7 & $\checkmark$ & – & Benzenoids: Phenols \\ 
 &  &  &  &  \\ 
\hspace*{0.2cm}\textbf{Combustion Byproducts} &  &  &  &  \\ 
\hspace*{0.4cm}\textit{\underline{Chlorinated}} &  &  &  &  \\ 
\hspace*{0.6cm}1,2,3,7,8-Pentachlorodibenzofuran & 57117-41-6 & $\checkmark$ & 3 & Organoheterocyclic compounds: Benzofurans \\ 
\hspace*{0.6cm}Octachlorodibenzo-p-dioxin & 3268-87-9 & $\checkmark$ & – & Organoheterocyclic compounds: Benzodioxins \\ 
\hspace*{0.6cm}Octachlorodibenzofuran & 39001-02-0 & $\checkmark$ & 3 & Organoheterocyclic compounds: Benzofurans \\ 
\hspace*{0.4cm}\textit{\underline{Polycyclic Aromatic Hydrocarbon}} &  &  &  &  \\ 
\hspace*{0.6cm}2-Methylnaphthalene & 91-57-6 & – & – & Benzenoids: Naphthalenes \\ 
\hspace*{0.6cm}3-Methylcholanthrene & 56-49-5 & $\checkmark$ & – & Benzenoids: Phenanthrenes and derivatives \\ 
\hspace*{0.6cm}7,12-Dimethylbenz(a)anthracene & 57-97-6 & $\checkmark$ & – & Benzenoids: Phenanthrenes and derivatives \\ 
\hspace*{0.6cm}Acenaphthene & 83-32-9 & – & 3 & Benzenoids: Naphthalenes \\ 
\hspace*{0.6cm}Acenaphthylene & 208-96-8 & $\checkmark$ & 2A & Benzenoids: Acenaphthylenes \\ 
\hspace*{0.6cm}Anthracene & 120-12-7 & – & 2B & Benzenoids: Anthracenes \\ 
\hspace*{0.6cm}Benz(a)anthracene & 56-55-3 & $\checkmark$ & 2B & Benzenoids: Phenanthrenes and derivatives \\ 
\hspace*{0.6cm}Benzo(b)fluoranthene & 205-99-2 & $\checkmark$ & 2B & Benzenoids: Phenanthrenes and derivatives \\ 
\hspace*{0.6cm}Benzo(g,h,i)perylene & 191-24-2 & – & 3 & Benzenoids: Pyrenes \\ 
\hspace*{0.6cm}Benzo(k)fluoranthene & 207-08-9 & $\checkmark$ & 2B & Benzenoids: Naphthalenes \\ 
\hspace*{0.6cm}Benzo[a]pyrene & 50-32-8 & $\checkmark$ & 1 & Benzenoids: Pyrenes \\ 
\hspace*{0.6cm}Chrysene & 218-01-9 & – & 2B & Benzenoids: Phenanthrenes and derivatives \\ 
\hspace*{0.6cm}Dibenz(a,h)anthracene & 53-70-3 & $\checkmark$ & 2A & Benzenoids: Phenanthrenes and derivatives \\ 
\hspace*{0.6cm}Fluoranthene & 206-44-0 & $\checkmark$ & 3 & Benzenoids: Naphthalenes \\ 
\hspace*{0.6cm}Fluorene & 86-73-7 & – & 3 & Benzenoids: Fluorenes \\ 
\hspace*{0.6cm}Indeno(1,2,3-cd)pyrene & 193-39-5 & – & 2B & Benzenoids: Pyrenes \\ 
\hspace*{0.6cm}Naphthalene & 91-20-3 & – & 2B & Benzenoids: Naphthalenes \\ 
\hspace*{0.6cm}Phenanthrene & 85-01-8 & – & 3 & Benzenoids: Phenanthrenes and derivatives \\ 
\hspace*{0.6cm}Pyrene & 129-00-0 & – & 3 & Benzenoids: Pyrenes \\ 
 &  &  &  &  \\ 
\hspace*{0.2cm}\textbf{Disinfectant Breakdown Products} &  &  &  &  \\ 
\hspace*{0.6cm}Bromodichloromethane & 75-27-4 & $\checkmark$ & 2B & Organohalogen compounds: Alkyl halides \\ 
\hspace*{0.6cm}Dibromochloromethane & 124-48-1 & – & 3 & Organohalogen compounds: Alkyl halides \\ 
 &  &  &  &  \\ 
\hspace*{0.2cm}\textbf{Dye intermediates} &  &  &  &  \\ 
\hspace*{0.6cm}1-Chloro-2-nitrobenzene & 88-73-3 & – & 2B & Benzenoids: Benzene and substituted derivatives \\ 
\hspace*{0.6cm}1-Chloro-4-nitrobenzene & 100-00-5 & – & 2B & Benzenoids: Benzene and substituted derivatives \\ 
\hspace*{0.6cm}1-Naphthol & 90-15-3 & – & – & Benzenoids: Naphthalenes \\ 
\hspace*{0.6cm}1-Naphthylamine & 134-32-7 & – & 3 & Benzenoids: Naphthalenes \\ 
\hspace*{0.6cm}2,4,5-Trimethylaniline & 137-17-7 & – & 3 & Benzenoids: Benzene and substituted derivatives \\ 
\hspace*{0.6cm}2,4-Diaminoanisole (as sulfate hydrate) & 615-05-4 & – & 2B & Benzenoids: Phenol ethers \\ 
\hspace*{0.6cm}2,4-Diaminotoluene & 95-80-7 & – & 2B & Benzenoids: Benzene and substituted derivatives \\ 
\hspace*{0.6cm}2-Aminobiphenyl & 90-41-5 & – & 1 & Benzenoids: Benzene and substituted derivatives \\ 
\hspace*{0.6cm}2-Chloro-6-nitrotoluene & 83-42-1 & – & – & Benzenoids: Benzene and substituted derivatives \\ 
\hspace*{0.6cm}2-Naphthylamine & 91-59-8 & – & 1 & Benzenoids: Naphthalenes \\ 
\hspace*{0.6cm}2-Nitroaniline & 88-74-4 & – & – & Benzenoids: Benzene and substituted derivatives \\ 
\hspace*{0.6cm}2-Phenylphenol & 90-43-7 & $\checkmark$ & 3 & Benzenoids: Benzene and substituted derivatives \\ 
\hspace*{0.6cm}3,3'-Dimethoxybenzidine & 119-90-4 & $\checkmark$ & 2B & Benzenoids: Benzene and substituted derivatives \\ 
\hspace*{0.6cm}3,3'-Dimethyl-4,4'-diaminodiphenylmethane & 838-88-0 & – & 2B & Benzenoids: Benzene and substituted derivatives \\ 
\hspace*{0.6cm}3,3'-Dimethylbenzidine & 119-93-7 & $\checkmark$ & 2B & Benzenoids: Benzene and substituted derivatives \\ 
\hspace*{0.6cm}3,4-Dichloroaniline & 95-76-1 & $\checkmark$ & – & Benzenoids: Benzene and substituted derivatives \\ 
\hspace*{0.6cm}3-Chloro-o-toluidine & 87-60-5 & – & – & Benzenoids: Benzene and substituted derivatives \\ 
\hspace*{0.6cm}3-Nitroaniline & 99-09-2 & – & – & Benzenoids: Benzene and substituted derivatives \\ 
\hspace*{0.6cm}4,4'-Thiodianiline & 139-65-1 & – & 2B & Organosulfur compounds: Thioethers \\ 
\hspace*{0.6cm}4-Aminobiphenyl & 92-67-1 & – & 1 & Benzenoids: Benzene and substituted derivatives \\ 
\hspace*{0.6cm}4-Chloro-2-nitrotoluene & 89-59-8 & – & – & Benzenoids: Benzene and substituted derivatives \\ 
\hspace*{0.6cm}4-Chloro-3-nitrotoluene & 89-60-1 & – & – & Benzenoids: Benzene and substituted derivatives \\ 
\hspace*{0.6cm}4-Chloroaniline & 106-47-8 & – & 2B & Benzenoids: Benzene and substituted derivatives \\ 
\hspace*{0.6cm}4-Nitroaniline & 100-01-6 & – & – & Benzenoids: Benzene and substituted derivatives \\ 
\hspace*{0.6cm}5-Nitro-o-toluidine & 99-55-8 & – & 3 & Benzenoids: Benzene and substituted derivatives \\ 
\hspace*{0.6cm}Aniline & 62-53-3 & $\checkmark$ & 2A & Benzenoids: Benzene and substituted derivatives \\ 
\hspace*{0.6cm}Anthraquinone & 84-65-1 & – & 2B & Benzenoids: Anthracenes \\ 
\hspace*{0.6cm}Azobenzene & 103-33-3 & – & 3 & Organoheterocyclic compounds: Azobenzenes \\ 
\hspace*{0.6cm}Benzidine & 92-87-5 & – & 1 & Benzenoids: Benzene and substituted derivatives \\ 
\hspace*{0.6cm}Carbazole & 86-74-8 & – & 2B & Organoheterocyclic compounds: Indoles and derivatives \\ 
\hspace*{0.6cm}o-Aminoazotoluene & 97-56-3 & $\checkmark$ & 2B & Organoheterocyclic compounds: Azobenzenes \\ 
\hspace*{0.6cm}o-Anisidine & 90-04-0 & – & 2A & Benzenoids: Phenol ethers \\ 
\hspace*{0.6cm}o-Toluidine & 95-53-4 & – & 1 & Benzenoids: Benzene and substituted derivatives \\ 
\hspace*{0.6cm}p-Aminoazobenzene & 60-09-3 & – & 2B & Organoheterocyclic compounds: Azobenzenes \\ 
\hspace*{0.6cm}p-Cresidine & 120-71-8 & – & 2B & Benzenoids: Phenol ethers \\ 
\hspace*{0.6cm}p-Dimethylaminoazobenzene & 60-11-7 & – & 2B & Organoheterocyclic compounds: Azobenzenes \\ 
\hspace*{0.6cm}p-Phenylenediamine & 106-50-3 & $\checkmark$ & 3 & Benzenoids: Benzene and substituted derivatives \\ 
 &  &  &  &  \\ 
\hspace*{0.2cm}\textbf{Explosive Intermediates} &  &  &  &  \\ 
\hspace*{0.6cm}2,4-Dinitrotoluene & 121-14-2 & $\checkmark$ & 2B & Benzenoids: Benzene and substituted derivatives \\ 
\hspace*{0.6cm}2,6-Dinitrotoluene & 606-20-2 & – & 2B & Benzenoids: Benzene and substituted derivatives \\ 
\hspace*{0.6cm}m-Dinitrobenzene & 99-65-0 & – & – & Benzenoids: Benzene and substituted derivatives \\ 
 &  &  &  &  \\ 
\hspace*{0.2cm}\textbf{Flame Retardants} &  &  &  &  \\ 
\hspace*{0.4cm}\textit{\underline{Brominated}} &  &  &  &  \\ 
\hspace*{0.6cm}2-Bromoallyl-2,4,6-tribromophenyl ether & 99717-56-3 & $\checkmark$ & – & Benzenoids: Phenol ethers \\ 
\hspace*{0.6cm}BDE-100 & 189084-64-8 & – & – & Benzenoids: Benzene and substituted derivatives \\ 
\hspace*{0.6cm}BDE-47 & 5436-43-1 & $\checkmark$ & – & Benzenoids: Benzene and substituted derivatives \\ 
\hspace*{0.6cm}BDE-66 & 189084-61-5 & – & – & Benzenoids: Benzene and substituted derivatives \\ 
\hspace*{0.6cm}Tetrabromobisphenol A & 79-94-7 & $\checkmark$ & 2A & Benzenoids: Benzene and substituted derivatives \\ 
\hspace*{0.6cm}Tetrabromobisphenol A bis(allyl ether) & 25327-89-3 & $\checkmark$ & – & Benzenoids: Benzene and substituted derivatives \\ 
\hspace*{0.6cm}Tetrabromobisphenol S & 39635-79-5 & $\checkmark$ & – & Benzenoids: Benzene and substituted derivatives \\ 
\hspace*{0.6cm}Tris(tribromoneopentyl) & 21850-44-2 & $\checkmark$ & – & Benzenoids: Benzene and substituted derivatives \\ 
\hspace*{0.6cm}Tris(tribromoneopentyl) phosphate & 19186-97-1 & – & – & Organic acids and derivatives: Organic phosphoric acids and derivatives \\ 
\hspace*{0.4cm}\textit{\underline{Chlorinated}} &  &  &  &  \\ 
\hspace*{0.6cm}Tri(Chloropropyl) phosphate & 1067-98-7 & – & – & Organic acids and derivatives: Organic phosphoric acids and derivatives \\ 
\hspace*{0.6cm}Tris(1,3-dichloro-2-propyl) phosphate & 13674-87-8 & $\checkmark$ & – & Organic acids and derivatives: Organic phosphoric acids and derivatives \\ 
\hspace*{0.6cm}Tris(1-chloro-2-propyl) phosphate & 13674-84-5 & $\checkmark$ & – & Organic acids and derivatives: Organic phosphoric acids and derivatives \\ 
\hspace*{0.6cm}Tris(2-chloroethyl) phosphate & 115-96-8 & $\checkmark$ & 3 & Organic acids and derivatives: Organic phosphoric acids and derivatives \\ 
\hspace*{0.4cm}\textit{\underline{Organophosphate}} &  &  &  &  \\ 
\hspace*{0.6cm}(3-Methylphenyl) diphenyl phosphate & 69500-28-3 & – & – & Organic acids and derivatives: Organic phosphoric acids and derivatives \\ 
\hspace*{0.6cm}Tri-M-cresyl phosphate & 563-04-2 & $\checkmark$ & – & Organic acids and derivatives: Organic phosphoric acids and derivatives \\ 
\hspace*{0.6cm}Tri-P-cresyl phosphate & 78-32-0 & $\checkmark$ & – & Organic acids and derivatives: Organic phosphoric acids and derivatives \\ 
\hspace*{0.6cm}Tripropyl Phosphate & 513-08-6 & – & – & Organic acids and derivatives: Organic phosphoric acids and derivatives \\ 
 &  &  &  &  \\ 
\hspace*{0.2cm}\textbf{Lubricants} &  &  &  &  \\ 
\hspace*{0.6cm}n-Docosane & 629-97-0 & – & – & Hydrocarbons: Saturated hydrocarbons \\ 
\hspace*{0.6cm}n-Dotriacontane & 544-85-4 & – & – & Hydrocarbons: Saturated hydrocarbons \\ 
\hspace*{0.6cm}n-Eicosane & 112-95-8 & – & – & Hydrocarbons: Saturated hydrocarbons \\ 
\hspace*{0.6cm}n-Heptadecane & 629-78-7 & – & – & Hydrocarbons: Saturated hydrocarbons \\ 
\hspace*{0.6cm}n-Hexacosane & 630-01-3 & – & – & Hydrocarbons: Saturated hydrocarbons \\ 
\hspace*{0.6cm}n-Octacosane & 630-02-4 & – & – & Hydrocarbons: Saturated hydrocarbons \\ 
\hspace*{0.6cm}n-Tetracosane & 646-31-1 & – & – & Hydrocarbons: Saturated hydrocarbons \\ 
\hspace*{0.6cm}n-Triacontane & 638-68-6 & – & – & Hydrocarbons: Saturated hydrocarbons \\ 
 &  &  &  &  \\ 
\hspace*{0.2cm}\textbf{Per- and Polyfluoroalkyl Substances (PFAS)} &  &  &  &  \\ 
\hspace*{0.6cm}9-Chlorohexadecafluoro-3-oxanone-1-sulfonic acid & 756426-58-1 & – & – & Organic acids and derivatives: Organic sulfonic acids and derivatives \\ 
\hspace*{0.6cm}N-methylperfluoro-1-octanesulfonamidoacetic acid (linear) & 2355-31-9 & – & – & Organohalogen compounds: Alkyl halides \\ 
\hspace*{0.6cm}Perfluoro-n-dodecanoic acid & 307-55-1 & $\checkmark$ & – & Organohalogen compounds: Alkyl halides \\ 
\hspace*{0.6cm}Perfluoro-n-tetradecanoic acid & 376-06-7 & – & – & Lipids and lipid-like molecules: Fatty Acyls \\ 
\hspace*{0.6cm}Perfluorohexanoic Acid & 307-24-4 & – & – & Organohalogen compounds: Alkyl halides \\ 
\hspace*{0.6cm}Perfluorononanoic acid & 375-95-1 & $\checkmark$ & – & Organohalogen compounds: Alkyl halides \\ 
 &  &  &  &  \\ 
\hspace*{0.2cm}\textbf{Plasticizers and Plastic Additives} &  &  &  &  \\ 
\hspace*{0.4cm}\textit{\underline{Plastic Additives}} &  &  &  &  \\ 
\hspace*{0.6cm}Bisphenol A & 80-05-7 & $\checkmark$ & – & Benzenoids: Benzene and substituted derivatives \\ 
\hspace*{0.6cm}Bisphenol F & 620-92-8 & $\checkmark$ & – & Benzenoids: Benzene and substituted derivatives \\ 
\hspace*{0.6cm}Bisphenol S & 80-09-1 & $\checkmark$ & – & Benzenoids: Benzene and substituted derivatives \\ 
\hspace*{0.6cm}N-Nitrosodiphenylamine & 86-30-6 & – & 3 & Benzenoids: Benzene and substituted derivatives \\ 
\hspace*{0.4cm}\textit{\underline{Plasticizer Metabolites}} &  &  &  &  \\ 
\hspace*{0.6cm}Mono(2-ethyl-5-hydroxyhexyl) phthalate & 40321-99-1 & – & – & Benzenoids: Benzene and substituted derivatives \\ 
\hspace*{0.6cm}Mono(2-ethyl-5-oxohexyl) phthalate & 40321-98-0 & – & – & Benzenoids: Benzene and substituted derivatives \\ 
\hspace*{0.6cm}Mono(5-carboxy-2-ethylpentyl) phthalate & 40809-41-4 & – & – & Benzenoids: Benzene and substituted derivatives \\ 
\hspace*{0.6cm}Mono-2-ethylhexyl phthalate & 4376-20-9 & $\checkmark$ & – & Benzenoids: Benzene and substituted derivatives \\ 
\hspace*{0.6cm}Mono-benzyl phthalate & 2528-16-7 & – & – & Benzenoids: Benzene and substituted derivatives \\ 
\hspace*{0.6cm}Mono-cyclohexyl phthalate & 7517-36-4 & – & – & Benzenoids: Benzene and substituted derivatives \\ 
\hspace*{0.6cm}Mono-ethyl phthalate & 2306-33-4 & – & – & Benzenoids: Benzene and substituted derivatives \\ 
\hspace*{0.6cm}Mono-isobutyl phthalate & 30833-53-5 & – & – & Benzenoids: Benzene and substituted derivatives \\ 
\hspace*{0.6cm}Mono-isononyl phthalate & 106610-61-1 & – & – & Benzenoids: Benzene and substituted derivatives \\ 
\hspace*{0.6cm}Mono-methyl phthalate & 4376-18-5 & – & – & Benzenoids: Benzene and substituted derivatives \\ 
\hspace*{0.6cm}Mono-n-butyl phthalate & 131-70-4 & $\checkmark$ & – & Benzenoids: Benzene and substituted derivatives \\ 
\hspace*{0.6cm}Mono-n-octyl phthalate & 5393-19-1 & $\checkmark$ & – & Benzenoids: Benzene and substituted derivatives \\ 
\hspace*{0.4cm}\textit{\underline{Plasticizers}} &  &  &  &  \\ 
\hspace*{0.6cm}2-Ethylhexyldiphenylphosphate & 1241-94-7 & $\checkmark$ & – & Organic acids and derivatives: Organic phosphoric acids and derivatives \\ 
\hspace*{0.6cm}Butyl benzyl phthalate & 85-68-7 & $\checkmark$ & 3 & Benzenoids: Benzene and substituted derivatives \\ 
\hspace*{0.6cm}Di-n-butyl phthalate & 84-74-2 & $\checkmark$ & – & Benzenoids: Benzene and substituted derivatives \\ 
\hspace*{0.6cm}Di-n-octyl phthalate & 117-84-0 & $\checkmark$ & – & Benzenoids: Benzene and substituted derivatives \\ 
\hspace*{0.6cm}Diethyl phthalate & 84-66-2 & $\checkmark$ & – & Benzenoids: Benzene and substituted derivatives \\ 
\hspace*{0.6cm}Dimethyl phthalate & 131-11-3 & $\checkmark$ & – & Benzenoids: Benzene and substituted derivatives \\ 
\hspace*{0.6cm}Tributyl Phosphate & 126-73-8 & $\checkmark$ & – & Organic acids and derivatives: Organic phosphoric acids and derivatives \\ 
\hspace*{0.6cm}Tripentyl Phosphate & 2528-38-3 & – & – & Organic acids and derivatives: Organic phosphoric acids and derivatives \\ 
\hspace*{0.6cm}Tris(2-butoxyethyl) phosphate & 78-51-3 & $\checkmark$ & – & Organic acids and derivatives: Organic phosphoric acids and derivatives \\ 
\hspace*{0.6cm}Tris(2-ethylhexyl) phosphate & 78-42-2 & – & – & Organic acids and derivatives: Organic phosphoric acids and derivatives \\ 
\hspace*{0.6cm}bis(2-Ethylhexyl)phthalate & 117-81-7 & $\checkmark$ & 2B & Benzenoids: Benzene and substituted derivatives \\ 
 &  &  &  &  \\ 
\hspace*{0.2cm}\textbf{Polychlorinated Biphenyls (PCBs)} &  &  &  &  \\ 
\hspace*{0.6cm}PCB-105 & 32598-14-4 & $\checkmark$ & 1 & Benzenoids: Benzene and substituted derivatives \\ 
\hspace*{0.6cm}PCB-118 & 31508-00-6 & $\checkmark$ & 1 & Benzenoids: Benzene and substituted derivatives \\ 
\hspace*{0.6cm}PCB-132 & 38380-05-1 & $\checkmark$ & 1 & Benzenoids: Benzene and substituted derivatives \\ 
\hspace*{0.6cm}PCB-138 & 35065-28-2 & $\checkmark$ & 1 & Benzenoids: Benzene and substituted derivatives \\ 
\hspace*{0.6cm}PCB-146 & 51908-16-8 & $\checkmark$ & 1 & Benzenoids: Benzene and substituted derivatives \\ 
\hspace*{0.6cm}PCB-153 & 35065-27-1 & $\checkmark$ & 1 & Benzenoids: Benzene and substituted derivatives \\ 
\hspace*{0.6cm}PCB-172 & 52663-74-8 & $\checkmark$ & 1 & Benzenoids: Benzene and substituted derivatives \\ 
\hspace*{0.6cm}PCB-18 & 37680-65-2 & $\checkmark$ & 1 & Benzenoids: Benzene and substituted derivatives \\ 
\hspace*{0.6cm}PCB-180 & 35065-29-3 & $\checkmark$ & 1 & Benzenoids: Benzene and substituted derivatives \\ 
\hspace*{0.6cm}PCB-187 & 52663-68-0 & $\checkmark$ & 1 & Benzenoids: Benzene and substituted derivatives \\ 
\hspace*{0.6cm}PCB-191 & 74472-50-7 & $\checkmark$ & 1 & Benzenoids: Benzene and substituted derivatives \\ 
\hspace*{0.6cm}PCB-66 & 32598-10-0 & $\checkmark$ & 1 & Benzenoids: Benzene and substituted derivatives \\ 
\hspace*{0.6cm}PCB-70 & 32598-11-1 & – & 1 & Benzenoids: Benzene and substituted derivatives \\ 
 &  &  &  &  \\ 
\hspace*{0.2cm}\textbf{Side-Reaction Byproducts} &  &  &  &  \\ 
\hspace*{0.6cm}N-Nitrosodi-n-propylamine & 621-64-7 & – & 2B & Organic nitrogen compounds: Organonitrogen compounds \\ 
 &  &  &  &  \\ 
\hspace*{0.2cm}\textbf{Solvents} &  &  &  &  \\ 
\hspace*{0.6cm}1,2,4-Trimethylbenzene & 95-63-6 & – & – & Benzenoids: Benzene and substituted derivatives \\ 
\hspace*{0.6cm}1,3-Dichloropropane & 142-28-9 & – & 2B & Organohalogen compounds: Organochlorides \\ 
\hspace*{0.6cm}2-Chlorotoluene & 95-49-8 & – & – & Benzenoids: Benzene and substituted derivatives \\ 
\hspace*{0.6cm}2-Picoline & 109-06-8 & – & 3 & Organoheterocyclic compounds: Pyridines and derivatives \\ 
\hspace*{0.6cm}4-Chlorotoluene & 106-43-4 & – & – & Benzenoids: Benzene and substituted derivatives \\ 
\hspace*{0.6cm}Ethyl acetate & 141-78-6 & – & – & Organic acids and derivatives: Carboxylic acids and derivatives \\ 
\hspace*{0.6cm}Ethylbenzene & 100-41-4 & $\checkmark$ & 2B & Benzenoids: Benzene and substituted derivatives \\ 
\hspace*{0.6cm}Isophorone & 78-59-1 & – & 2B & Organic oxygen compounds: Organooxygen compounds \\ 
\hspace*{0.6cm}N-Nitrosomorpholine & 59-89-2 & – & 2B & Organoheterocyclic compounds: Oxazinanes \\ 
\hspace*{0.6cm}Toluene & 108-88-3 & $\checkmark$ & 3 & Benzenoids: Benzene and substituted derivatives \\ 
\hspace*{0.6cm}bis(2-Chloro-1-methylethyl)ether & 108-60-1 & – & 3 & Organic oxygen compounds: Organooxygen compounds \\ 
\hspace*{0.6cm}bis(2-Chloroethoxy)methane & 111-91-1 & – & – & Organic oxygen compounds: Organooxygen compounds \\ 
\hspace*{0.6cm}n-Butylbenzene & 104-51-8 & – & – & Benzenoids: Benzene and substituted derivatives \\ 
\hspace*{0.6cm}n-Dodecane & 112-40-3 & – & – & Hydrocarbons: Saturated hydrocarbons \\ 
\hspace*{0.6cm}n-Hexadecane & 544-76-3 & – & – & Hydrocarbons: Saturated hydrocarbons \\ 
\hspace*{0.6cm}n-Octadecane & 593-45-3 & – & – & Hydrocarbons: Saturated hydrocarbons \\ 
\hspace*{0.6cm}n-Pentadecane & 629-62-9 & – & – & Hydrocarbons: Saturated hydrocarbons \\ 
\hspace*{0.6cm}n-Tetradecane & 629-59-4 & – & – & Hydrocarbons: Saturated hydrocarbons \\ 
\hspace*{0.6cm}n-Tridecane & 629-50-5 & – & – & Hydrocarbons: Saturated hydrocarbons \\ 
\hspace*{0.6cm}o-Xylene & 95-47-6 & – & 3 & Benzenoids: Benzene and substituted derivatives \\ 
\hspace*{0.6cm}p-Isopropyltoluene & 99-87-6 & – & – & Lipids and lipid-like molecules: Prenol lipids \\ 
\hspace*{0.6cm}sec-Butylbenzene & 135-98-8 & – & – & Benzenoids: Benzene and substituted derivatives \\ 
\hspace*{0.6cm}tert-Butylbenzene & 98-06-6 & – & – & Benzenoids: Benzene and substituted derivatives \\ 
 &  &  &  &  \\ 
\hspace*{0.2cm}\textbf{Surfactants or Detergents} &  &  &  &  \\ 
\hspace*{0.6cm}4-Tert-Octylphenol & 140-66-9 & $\checkmark$ & – & Benzenoids: Benzene and substituted derivatives \\ 
\hspace*{0.6cm}Nonylphenol & 104-40-5 & $\checkmark$ & – & Benzenoids: Phenols \\ 
\hspace*{0.6cm}Nonylphenol diethoxylate & 9016-45-9 & $\checkmark$ & – & Benzenoids: Phenol ethers \\ 
\hspace*{0.6cm}Nonylphenol monoethoxylate & 27986-36-3 & $\checkmark$ & – & Benzenoids: Phenol ethers \\ 
 &  &  &  &  \\ 
\hspace*{0.2cm}\textbf{Wood Preservatives} &  &  &  &  \\ 
\hspace*{0.6cm}2,4,5-Trichlorophenol & 95-95-4 & $\checkmark$ & 2B & Benzenoids: Phenols \\ 
\hspace*{0.6cm}2,4,6-Trichlorophenol & 88-06-2 & $\checkmark$ & 2B & Benzenoids: Phenols \\ 
\hspace*{0.6cm}Pentachloroanisole & 1825-21-4 & – & – & Benzenoids: Phenol ethers \\ 
\hspace*{0.6cm}Pentachlorobenzene & 608-93-5 & $\checkmark$ & – & Benzenoids: Benzene and substituted derivatives \\ 
\hspace*{0.6cm}Pentachlorophenol & 87-86-5 & $\checkmark$ & 1 & Benzenoids: Phenols \\ 
\bottomrule
\end{longtable}


\end{landscape}

\newpage

\markright{\textit{Supplementary Table 3}}

\begin{landscape}

\noindent{\fontsize{12}{14.4}\selectfont\textbf{SUPPLEMENTARY TABLE 3}}

\noindent{\fontsize{10}{12}\selectfont\textbf{Comparison of quantitative values in literature.} The geometric mean or mean values for all chemicals from literature with quantitative values. The `type' column indicates which type of biological matrix the quantitative value was collected from. The 'type detail' column specifies if there were specific descriptors provided with the values, such as: 1) smokers/nonsmokers, 2) high-exposure status, 3) adipose tissue depot, or 4) cohort. In some cases, the same compound had multiple quantitative values listed from different studies. The values listed in the table are the maximum values across all studies and cohorts. The values listed in the range columns indicate the range of values in both the literature and in the present study. The high end of the range indicates the maximum value. When the low end of the range is below the detection limits, the range is listed with a less than sign. When the low end of the range is missing or not listed in the literature, the low end of the range is the theoretical minimum value (i.e., one half of the lowest value actually detected).}

[INSERT ST3 HERE]

\end{landscape}

\newpage

\markright{\textit{Supplementary Table 4}}

\begin{landscape}

\noindent{\fontsize{12}{14.4}\selectfont\textbf{SUPPLEMENTARY TABLE 4}}

\noindent{\fontsize{10}{12}\selectfont\textbf{Quantitative estimates for all chemicals detected in both tumor and non-neoplastic thyroid tissue.} This table lists the estimated PPB values for all chemicals annotated in both tumor samples and non-neoplastic thyroid tissue. The estimated PPB values are calculated by taking the mean area for each chemical and dividing it by the mean area from reference standards with known concentrations (0·47 ng/mL). Chemicals are sorted alphabetically.}

[INSERT ST4 HERE]

\end{landscape}

\newpage

\markright{\textit{Table Abbreviation Dictionary}}

\noindent{\fontsize{12}{14.4}\selectfont\textbf{TABLE ABBREVIATION DICTIONARY}}

\noindent{\fontsize{10}{12}\selectfont The supplementary tables have a substantial number of abbreviations, largely for chemical names. A full dictionary of abbreviations relevant to all supplementary tables can be found below:}

\fontsize{10}{12}\selectfont
\begin{itemize}
\setlength{\itemsep}{2pt}
\setlength{\parskip}{0pt}
\setlength{\parsep}{0pt}
  \item B8-1413 = 2,3,5,6,8,8,10,10-Octachlorobornane
  \item B9-1025 = 2,2,5,5,8,9,9,10,10-Nonachlorobornane
  \item B9-1679 = 2,3,5,6,8,8,9,10,10-Nonachlorobornane
  \item BDE = Brominated diphenyl ether
  \item CAS = Chemical Abstracts Service (Number)
  \item CDC = Centers for Disease Control and Prevention
  \item DDD = Dichlorodiphenyldichloroethane
  \item DDE = Dichlorodiphenyldichloroethylene
  \item DDT = Dichlorodiphenyltrichloroethane
  \item DFTPP = Decafluorotriphenylphosphine
  \item EDC = endocrine-disrupting chemical
  \item EPN = Ethyl p-nitrophenyl phenylphosphorothioate
  \item HpCDD = Heptachlorodibenzo-p-dioxin
  \item HpCDF = Heptachlorodibenzofuran
  \item HxCDD = Hexachlorodibenzo-P-dioxin
  \item HxCDF = Hexachlorodibenzofuran
  \item IARC = International Agency for Research on Cancer
  \item NHANES = National Health and Nutrition Examination Survey
  \item NPE = Nitrophenyl ether
  \item PAH = polycyclic aromatic hydrocarbon
  \item PBB = Polybrominated biphenyl
  \item PCB = Polychlorinated biphenyl
  \item PPB = parts per billion
  \item PeCDD = Pentachlorodibenzo-p-dioxin
  \item RT = retention time
  \item TCP = Trichlorophenyl
  \item min. = minutes
  \item mz = mass-to-charge ratio
\end{itemize}


\newpage
