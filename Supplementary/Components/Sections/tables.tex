\newpage

\markright{\textit{Supplementary Table 1}}

\begin{landscape}

\noindent{\fontsize{12}{14.4}\selectfont\textbf{SUPPLEMENTARY TABLE 1}}

\noindent{\fontsize{10}{12}\selectfont\textbf{The full library of xenobiotic chemicals employed for chemical identification.} The library of 710 confirmed xenobiotic chemicals employed for chemical identification. All chemicals were present in pooled reference plasma at a concentration of 0·47 ng/mL. There are 710 total unique chemicals (i.e., unique CAS numbers), but for some chemicals, there are multiple fragments from different standards used for identification, thus resulting in 892 total rows in the table. The individual mz columns indicate typical fragments observed for the given chemical.}

[INSERT ST1 HERE - TO BE GENERATED PROGRAMMATICALLY]

\end{landscape}

\newpage

\markright{\textit{Supplementary Table 2}}

\begin{landscape}

\noindent{\fontsize{12}{14.4}\selectfont\textbf{SUPPLEMENTARY TABLE 2}}

\noindent{\fontsize{10}{12}\selectfont\textbf{Metadata for all chemicals detected in samples.} Both the long-form chemical name and alias or abbreviation are listed if there is sufficient space. However, for chemical names that are too long or redundant, only the alias or abbreviation has been listed. The column `variant diff.' specifies if the chemical had differential abundance or detection between the three variants.}

[INSERT ST2 HERE - TO BE GENERATED PROGRAMMATICALLY]

\end{landscape}

\newpage

\markright{\textit{Supplementary Table 3}}

\noindent{\fontsize{12}{14.4}\selectfont\textbf{SUPPLEMENTARY TABLE 3}}

\noindent{\fontsize{10}{12}\selectfont\textbf{Observed concentrations versus reported literature values.} All values originally published as ng/g, $\mu$g/L, ng/mL are listed as PPB. Values originally published as pg/mL were converted to ng/mL and then listed as PPB. All values are rounded to the nearest integer or are listed as $<$ 1 PPB when applicable. The corresponding reference from which the comparison value is derived is cited next to the listed concentration.}

[INSERT ST3 HERE - TO BE GENERATED PROGRAMMATICALLY]

\newpage

\markright{\textit{Supplementary Table 4}}

\noindent{\fontsize{12}{14.4}\selectfont\textbf{SUPPLEMENTARY TABLE 4}}

\noindent{\fontsize{10}{12}\selectfont\textbf{Quantitative estimates of chemicals in non-cancer thyroids and tumors.} Data table containing quantitative estimates of chemical concentrations in thyroid tissues.}

[INSERT ST4 HERE - TO BE GENERATED PROGRAMMATICALLY]

\newpage

\noindent{\fontsize{12}{14.4}\selectfont\textbf{TABLE ABBREVIATION DICTIONARY}}
\markright{\textit{Table Abbreviation Dictionary}}

\vspace{0.5em}

The supplementary tables have a substantial number of abbreviations, largely for chemical names. A full dictionary of abbreviations relevant to all supplementary tables can be found below:

\begin{itemize}
\item 9Cl-PF3ONS = 9-Chlorohexadecafluoro-3-oxanone-1-sulfonic acid
\item BDCPP = bis(1,3-Dichloro-2-propyl) phosphate
\item BDE = brominated diphenyl ether
\item BDPP = Bis(2,3-dibromopropyl) hydrogen phosphate
\item Bromo-TMP-Phenol = 2-Bromo-4-(2,4,4-trimethylpentan-2-yl)phenol
\item CAS = Chemical Abstracts Service (Number)
\item CDC = Centers for Disease Control and Prevention
\item CID = Compound ID (PubChem)
\item Compds. = compounds
\item DBahA = Dibenz(a,h)anthracene
\item DCP = Dichlorophenyl
\item DCPMNB  = 4-(2,4-dichlorophenoxy)-2-methyl-1-nitrobenzene
\item DDD = Dichlorodiphenyldichloroethane
\item DDE = Dichlorodiphenyldichloroethylene
\item DDT = Dichlorodiphenyltrichloroethane
\item DFTPP = Decafluorotriphenylphosphine
\item DTPAs = Dithiophosphoric Acids
\item EPN = Ethyl p-nitrophenyl phenylphosphorothioate
\item EtFOSAA = N-Ethylperfluoro-1-octanesulfonamidoacetic acid (linear)
\item Furaneol = 4-Hydroxy-2,5-dimethyl-3(2H)-furanone
\item HpCDD = Heptachlorodibenzo-p-dioxin
\item HpCDF = Heptachlorodibenzofuran
\item HxCDD = Hexachlorodibenzo-P-dioxin
\item HxCDF = Hexachlorodibenzofuran
\item IARC = International Agency for Research on Cancer
\item IMHP = 2-Isopropyl-6-methyl-4-pyrimidinol
\item Lin. = linear
\item LLMs = lipid-like molecules
\item LOD = limit of detection
\item MBOT = 4,4'-Methylenebis(o-toluidine)
\item MCPA = 2-Methyl-4-chlorophenoxyacetic acid
\item MEcPP = Mono(5-carboxy-2-ethylpentyl) phthalate
\item MEHHP = Mono(2-ethyl-5-hydroxyhexyl) phthalate
\item MEOHP = Mono(2-ethyl-5-oxohexyl) phthalate
\item MGK-264 = McLaughlin Gormley King-264 (also known as N-2-Ethylhexylbicycloheptenedicarboximide)
\item min = minutes
\item MOCA = 4,4'-Methylenebis(2-chloroaniline)
\item mz = mass-to-charge ratio
\item N-MeFOSAA = N-Methylperfluoro-1-octanesulfonamidoacetic acid (linear)
\item NHANES = National Health and Nutrition Examination Survey
\item NPE = nitrophenyl ether
\item o-Dianisidine = 3,3'-Dimethoxybenzidine
\item OD-PABA = Octyl-dimethyl-p-aminobenzoic acid
\item Org = organic
\item Org. Heterocycl. = organoheterocyclic
\item p-Chlorocresol = 4-Chloro-3-methylphenol
\item PAH = polycyclic aromatic hydrocarbon
\item PBB = polybrominated biphenyl
\item PCB = polychlorinated biphenyl
\item PCDF = Pentachlorodibenzofuran
\item PeCDD = pentachlorodibenzo-p-dioxin
\item PKs = polyketides
\item PPB = parts per billion
\item RT = retention time
\item SDs = Substituted Derivatives
\item TBBPA-BAE = Tetrabromobisphenol A bis(allyl ether)
\item TCDD = tetrachlorodibenzo-p-dioxin
\item TCDF = tetrachlorodibenzofuran
\item TCP = Trichlorophenyl
\item TCP-4'-NPE = TCP-4'-NPE
\item TCPP = Tris(1-chloro-2-propyl) phosphate
\item TDCPP = Tris(1,3-dichloro-2-propyl)phosphate
\item TEEP = Tetraethyl ethylenediphosphonate
\item TPAs = Thiophosphoric acids
\item TTBNPP = Tris(tribromoneopentyl) phosphate
\end{itemize}

\newpage

\noindent{\fontsize{12}{14.4}\selectfont\textbf{REFERENCES CITED IN SUPPLEMENTARY MATERIAL}}
\markright{\textit{References Cited in Supplementary Material}}
